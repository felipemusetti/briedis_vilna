\textbf{Vilna, cidade dos outros} é uma narrativa sobre a capital da Lituânia escrita a partir da cartografia histórica e geografia humana local, constituída ao longo do tempo por inúmeros povos falantes de diversos idiomas, como judeus, poloneses, lituanos, ucranianos, bielorussos, russos, alemães, letões, armênios, tártaros e outros grupos minoritários. Impregnada dentre seus vários componentes pelo barroco, que esteve no limiar da Europa e no contexto de suas mudanças, a cosmopolita cidade também é apresentada através de textos de pessoas ilustres ou desconhecidas, de muitas procedências e línguas, que viveram ou passaram por ela, em experiências, sensibilidades e perspectivas próprias.

\textbf{Laimonas Briedis} é nascido em Vilna mas mora atualmente em Vancouver, no Canadá. Tem doutorado em Geografia Humana pela University of British Columbia e pós-doutorado no Departamento de História da University of Toronto, e é professor no Seminário Literário de Verão em Vilna iniciado pela Concordia University de Montreal. Sua pesquisa é focada na complexa trama da história da Lituânia em relação à Europa e ao mundo. Acompanhou as várias diásporas da Lituânia por toda a Europa e América do Norte, além de Jerusalém e Xangai.

\textbf{Fernando Klabin} nasceu e cresceu em São Paulo. Formou-se em Ciência Política pela Universidade de Bucareste e, além de tradutor, exerce atividades ocasionais como fotógrafo, escritor, ator e artista plástico. Descende de uma família de \textit{litvaks} que migrou da Rússia para o Brasil no final do século \textsc{xix}.

\textbf{Celso Lafer} é advogado, jurista, professor titular do Departamento de Filosofia e Teoria Geral do Direito da \textsc{usp} e membro da Academia Brasileira de Letras. Assumiu diversos cargos públicos como Ministro de Estado das Relações Exteriores (1992 e 2001--2002), Ministro de Estado do Desenvolvimento, Indústria e Comércio (1999), Embaixador da Missão Permanente do Brasil junto às Nações Unidas e à Organização Mundial do Comércio (1995--1998), dentre outros. Descende de uma família judaica lituana, que imigrou para o Brasil no final do século \textsc{xix}. 