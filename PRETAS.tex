\textbf{Vilna, cidade dos outros} é uma narrativa sobre a capital da Lituânia. Escrita com base na cartografia histórica e geografia humana local, a cidade que também foi conhecida como \textit{a Jerusalém da Lituânia} abrigou ao longo dos tempos inúmeros povos, falantes de diversos idiomas, em uma miscelânia cultural: judeus, poloneses, lituanos, ucranianos, bielorussos, russos, alemães, letões, armênios, tártaros e outros grupos minoritários. Impregnada dentre seus vários componentes pelo barroco, que esteve no limiar da Europa e no contexto de suas mudanças, a cosmopolita cidade também é apresentada através de textos de pessoas ilustres ou desconhecidas, de muitas procedências e línguas, que viveram ou passaram por ela, através de relatos de experiências, sensibilidades e perspectivas próprias.

\textbf{Laimonas Briedis} é nascido em Vilna mas mora atualmente em Vancouver, no Canadá. Tem doutorado em Geografia Humana pela University of British Columbia e pós-doutorado no Departamento de História da University of Toronto, e é professor no Seminário Literário de Verão em Vilna iniciado pela Concordia University de Montreal. Sua pesquisa é focada na complexa trama da história da Lituânia em relação à Europa e ao mundo. Acompanhou as várias diásporas da Lituânia por toda a Europa e América do Norte, além de Jerusalém e Xangai.

\textbf{Fernando Klabin} nasceu e cresceu em São Paulo. Formou-se em Ciência Política pela Universidade de Bucareste e, além de tradutor, exerce atividades ocasionais como fotógrafo, escritor, ator e artista plástico. Descende de uma família de \textit{litvaks} que imigrou para o Brasil no final do século \textsc{xix}.

\textbf{Celso Lafer} é jurista e Professor Emérito da \textsc{usp} e de sua Faculdade de Direito, na qual lecionou até a sua aposentadoria em 2011. É também membro da Academia Brasileira de Letras e da Academia Paulista de Letras. Exerceu diversos cargos públicos como ministro de Estado das Relações Exteriores (1992 e 2001--2002), ministro de Estado do Desenvolvimento, Indústria e Comércio (1999), embaixador chefe da Missão Permanente do Brasil junto às Nações Unidas e à Organização Mundial do Comércio (1995--1998), dentre outros. Descende de uma família de \textit{litvaks} que imigrou para o Brasil no final do século \textsc{xix}.




