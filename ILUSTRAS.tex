\chapter{Ilustrações}

\begin{enumerate}
\def\labelenumi{\arabic{enumi}.}
\item
  Frontispício de S. Muenster, \emph{La cosmographie universelle},
  Basiléia, 1556. Biblioteca da Universidade de Vilnius.
\item
  Wilno. Fotografia de J. Bułhak, ca. 1939. Biblioteca da Academia
  Lituana de Ciências.
\item
  Vilnius em várias línguas. Detalhe de ``Lithuania'', preparado por M.
  V. Coronelli, Veneza, 1696. Biblioteca da Universidade de Vilnius.
\item
  ``Novo mapa da Europa'', de S. Muenster, \emph{La cosmographie
  universelle}, Basiléia, 1556. Biblioteca da Universidade de Vilnius.
\item
  Gediminas constrói um castelo no lugar de seu sonho. Xilogravura de M.
  E. Andriolli e B. Puc, 1882. Museu Lituano de Arte.
\item
  Lituanos pagãos venerando fogo, o carvalho e a serpente do jardim, de
  S. Muenster, \emph{La cosmographie universelle}, Basiléia, 1556.
  Biblioteca da Universidade de Vilnius.
\item
  Cavaleiro teutônico, de S. Muenster, \emph{La cosmographie
  universelle}, Basiléia, 1556. Biblioteca da Universidade de Vilnius.
\item
  ``Lithuania'' de H. Schedel, \emph{Weltchronik}, Nurembergue, 1493.
  Biblioteca da Universidade de Vilnius.
\item
  Ostra Brama. Litografia de J. Hoppen, 1924. Biblioteca da Universidade
  de Vilnius.
\item
  São Casimiro. Gravura em cobre de F. Balcewicz, 1749. Biblioteca da
  Universidade de Vilnius.
\item
  ``Sarmácia, província limiar da Europa'', de S. Muenster,
  \emph{Cosmographiae Universalis}, Basiléia, 1572. Coleção particular
  de L. Briedis.
\item
  Alce lituano, de S. Muenster, \emph{La cosmographie universelle},
  Basiléia, 1556. Biblioteca da Universidade de Vilnius.
\item
  A Sarmácia na Europa e Ásia, de C. Ptolemy, \emph{Geographia},
  Estrasburgo, 1513. Biblioteca da Universidade de Vilnius.
\item
  Frontispício de A. Gwagnini, \emph{Sarmatiae Europeae
  Descriptio}\ldots{}, Cracóvia, 1578. Biblioteca da Universidade de
  Vilnius.
\item
  ``Wilna ou Wilda, capital da Lituânia'', G. Bodenehr, Augsburgo, ca.
  1720. Biblioteca da Universidade de Vilnius.
\item
  São Cristóvão, de \emph{Staroe Vilno}, ca. 1906. Biblioteca da
  Universidade de Vilnius.
\item
  A ressurreição de Vilnius. Frontispício de A. Pozniak, \emph{Senator
  septem consulabris}\ldots{}, Vilnae, 1666. Biblioteca da Universidade
  de Vilnius.
\item
  O triunfo da Polônia. Ilustração de N. Kiszka De Ciechanowiec,
  \emph{Triumphale Solium Serenissimae Regine Poloniarum}\ldots{},
  Vilnae, 1637. Biblioteca da Universidade de Vilnius.
\item
  ``A peste em Vilnius em 1710''. Gravura em cobre de X. Karęga a partir
  de um desenho de F. Pelikan, 1799. Museu Lituano de Arte.
\item
  Estradas para Vilnius. Detalhe de ``Neueste Karte von Polen und
  Litauen'', F. Müller e C. Schütz, Viena, 1792. Biblioteca da
  Universidade de Vilnius.
\item
  ``Mapa de Willda ou Willna na Lituânia'', autor anônimo, 1737. Museu
  Lituano de Arte.
\item
  O grande pátio da universidade em Wilna. Desenho de F. Smuglewicz,
  1786. Biblioteca da Universidade de Vilnius.
\item
  Comerciantes judeus perto de Wilna. Cromolitografia de L. Bichebois,
  I. Deroy e K. Kukiewicz, retirado de J.\,L. Wilczyński, \emph{Album de
  Wilna}, Paris, 1848. Biblioteca da Universidade de Vilnius.
\item
  A muralha da cidade de Wilna. Desenho de F. Smuglewicz, 1785. Museu
  Lituano de Arte.
\item
  Vista do jardim botânico da universidade. Aquarela de J. Pezska, 1808.
  Biblioteca da Universidade de Vilnius.
\item
  Apagando a Sarmácia do mapa da Europa. ``A Divisão da
  Polônia"-Lituânia'', gravura de J.\,E. Nielsen, 1773. Museu Nacional
  Lituano.
\item
  Vilna: o cordão da cidade. Litografia de K. Bachmatowicz retirada de
  \emph{Przypomniene Wilna} (\emph{Memórias de Wilna}), 1837. Biblioteca
  da Universidade de Vilnius.
\item
  A Universidade Imperial de Vilna na primeira metade do século 19.
  Cromolitografia de P. Benoist e A. Bayot, retirada de J.\,K.
  Wilczyński, \emph{Album de Wilna}, Paris, 1850. Museu Lituano de Arte.
\item
  Cena de rua em Vilna. Litografia de K. Bachmatowicz retirada de
  \emph{Przypomniene Wilna} (\emph{Memórias de Wilna}), 1837. Biblioteca
  da Universidade de Vilnius.
\item
  A \emph{Grande Armée} cruzando o rio Niemen (Nemunas) em 1812.
  Litografia de I. Klauber baseada numa pintura de Bagetti, impressa em
  São Petersburgo, ca. 1850. Museu Nacional Lituano.
\item
  Mapa da Polônia e Lituânia. Detalhe de ``Regni Poloniae, Magni Ducatus
  Lituaniae\ldots{}'', J.\,J. Kanter, folha 7, Regiomonti, 1770.
  Biblioteca da Universidade de Vilnius.
\item
  Palácio do governador"-geral em Vilna. Cromolitografia de P. Benoist,
  retirada de J.\,K. Wilczyński, \emph{Album de Wilna}, Paris, 1850.
  Museu Lituano de Arte.
\item
  Gráfico representando o colapso do exército napoleônico durante a
  campanha russa de 1812, de C.\,J. Minard, 1869. Coleção particular de
  L. Briedis.
\item
  Oficiais franceses em Vilna salvos por um monge samaritano das mãos de
  assaltantes locais. ``Assalto em De Bissy'', litografia de J.
  Oziebłowski, 1844. Museu Lituano de Arte.
\item
  A retirada da \emph{Grande Armée} por Vilna em 1812. Litografia de V.
  Adam e L. Bichebois baseada numa pintura de J. Damel, retirada de J.
  K. Wilczyński, \emph{Album de Wilna}, Paris, 1846. Biblioteca da
  Universidade de Vilnius.
\item
  Alexandre \versal{I} passa em revista as tropas após a captura de Vilna por
  parte das forças russas em dezembro de 1812. Desenho de A. Chicherin,
  retirado de \emph{Dnevnik Aleksandra Chicherina}, Moscou: Nauka, 1966.
  Biblioteca da Universidade de Vilnius.
\item
  Vista de Vilna a partir das colinas circundantes na década de 1820.
  Litografia de J. Hoppen, 1925. Museu Lituano de Arte.
\item
  Rua Ostrabrama em Vilna na primeira metade do século 19.
  Cromolitografia de L. Bichebois e V. Adam baseada numa pintura de M.
  Zaleski, retirada de J.\,K. Wilczyński, \emph{Album de Wilna}, Paris,
  1846. Biblioteca da Universidade de Vilnius.
\item
  Mapa da ferrovia São Petersburgo -- Varsóvia, retirado de A.\,H. Kirkor,
  \emph{Przewodnik: Wilno}, Wilna, 1863. Biblioteca da Academia Lituana
  de Ciências.
\item
  A Igreja Católica de São Casimiro em Vilna se transformou na Catedral
  Ortodoxa Russa de São Nicolau depois da insurreição polono"-lituana de
  1863--1864 contra o domínio tzarista. Foto de S.\,F. Fleury, ca. 1896.
  Museu Nacional Lituano.
\item
  Entrada do pátio da Velha Sinagoga de Vilne, ca. 1900. Cartão"-postal,
  cortesia de A. Kubilas.
\item
  Mapa de Vilna, 1882. Museu Nacional Lituano.
\item
  Interior da Igreja de Nosso Senhor Jesus (trinitária). Cromolitografia
  de I. Deroy baseada em trabalho de V. Sadovnikov, retirada de J.\,K.
  Wilczyński, \emph{Album de Wilna}, Paris, 1847. Museu Lituano de Arte.
\item
  Vilna como antigo portão de entrada para o Império Russo, 1872.
  Frontispício de P.\,N. Batyushkov, \emph{Pamiatniki Russkoi stariny},
  Vilna, 1872. Biblioteca da Academia Lituana de Ciências.
\item
  ``Saudações de Vilna: Rua Bolshaya {[}Grande{]}'', ca. 1900.
  Cartão"-postal, cortesia de A. Kubilas.
\item
  Nadando no Rio Viliya. Fotografia de S.\,F. Fleury, 1900. Biblioteca da
  Academia Lituana de Ciências.
\item
  Rua em Vilna; cartão"-postal a partir de um desenho de M. Dobuzhinsky,
  ca. 1914. Cartão"-postal, cortesia de A. Kubilas.
\item
  Capela de São Casimiro em Vilna; cartão"-postal a partir de uma
  fotografia de J. Bułhak, ca. 1910. Cartão"-postal, cortesia de A.
  Kubilas.
\item
  Vilna russa, ca. 1900. Cartão"-postal, cortesia de A. Kubilas.
\item
  Mapa estatístico da Lituânia, concebido pela força de ocupação alemã a
  fim de evidenciar a divisão étnica no país, intitulado
  ``Verwaltungsbezirk der Militärverwaltung Litauen'', 1918. Biblioteca
  da Universidade de Vilnius.
\item
  ``A captura de Wilna, cidade de governo russo, agosto de 1915''.
  Cartão"-postal, cortesia de A. Kubilas.
\item
  Wilna: panorama desde a Colina do Castelo, 1916. Cartão"-postal,
  cortesia de A. Kubilas.
\item
  Soldado"-flâneur alemão em Wilna, 1916. Cartão"-postal, cortesia de A.
  Kubilas.
\item
  Mapa de Wilna e cercanias com destaque às vias principais que cruzam a
  cidade, intitulado ``Garnison"-Umgebungskarte von Wilna'', 1917. Museu
  Nacional Lituano.
\item
  Saguão de espera da estação ferroviária de Wilna. Desenho de W. Buhe
  retirado de \emph{Bilderschau der Wilnaer Zeitung}, 3 de abril de
  1916. Biblioteca da Academia Lituana de Ciências.
\item
  Exposição dos Ateliês de Trabalho de Wilna. Cartaz de M. Bühlmann,
  1916. Biblioteca da Academia Lituana de Ciências.
\item
  Wilna: cena em tempo de guerra num mercado de pulgas movimentado, ca.
  1916. Cartão"-postal, cortesia de A. Kubilas.
\item
  ``Saudações natalinas de Wilna: recanto pitoresco'', 1916.
  Cartão"-postal, cortesia de A. Kubilas.
\item
  Dia de festa na Velha Sinagoga de Wilna. Desenho de W. Buhe retirado
  de \emph{Bilderschau der Wilnaer Zeitung}, 29 de março de 1916.
  Biblioteca da Academia Lituana de Ciências.
\item
  A mesquita de madeira de Wilna, 1916. Cartão"-postal, cortesia de A.
  Kubilas.
\item
  Wilna; quarto ano de guerra, 1917. Cartão"-postal, cortesia de A.
  Kubilas.
\item
  Mapa da Polônia e Lituânia, publicado por G. Freytag \& Berndt, Viena,
  ca. 1923. Biblioteca da Universidade de Vilnius.
\item
  Wilno: Avenida São Jorge, ca. 1920. Cortesia de A. Kubilas.
\item
  Wilno lê em várias línguas. Fotografia da \emph{The National
  Geographic Magazine} 74--6, junho de 1938, p. 779.
\item
  Portão Ostra Brama; fotografia de J. Bułhak, ca. 1920. Cartão"-postal,
  cortesia de A. Kubilas.
\item
  Mapa de Vilne em iídiche, ca. 1940, retirado de Leyzer Ran,
  \emph{Jerusalem of Lithuania}, New York, 1974. Biblioteca da
  Universidade de Vilnius.
\item
  Cena de rua no bairro judaico de Vilne, ca. 1925. Cartão"-postal,
  cortesia de A. Kubilas.
\item
  Vista panorâmica de Wilna a partir da Colina do Castelo, de J.
  Grutzka, publicado em \emph{Zeitung der 10. Armee}, 1917. Biblioteca
  da Universidade de Vilnius.
\item
  Velho cemitério judaico em Vilne, ca. 1920. Cartão"-postal, cortesia de
  A. Kubilas.
\item
  Vilnius no centro da Europa. Ilustração retirada de \emph{Vilnius:
  Unforgettable Harmony and Charm}, Divisão de Turismo do Departamento
  Econômico da Prefeitura de Vilnius, 2002. Reproduzido sob permissão da
  Prefeitura de Vilnius.
\item
  Wilno: Rua Zamkowa (do Castelo), ca. 1930. Cartão"-postal, cortesia de
  A. Kubilas.
\item
  Ruínas de Vilnius no pós"-guerra, com suas igrejas barrocas ilesas, ca.
  1947. Fotografia retirada de \emph{Tarybų Lietuva}, 1940--1950,
  Vilnius: Vaga, 1950. Biblioteca da Universidade de Vilnius.
\item
  Mapa turístico da Vilnius soviética. Ilustração retirada de A. Papšys,
  \emph{Vilnius: a guide}, Moscou: Progress Publishers, 1981, pp. 88--89.
\item
  Domingo de Ramos em Vilnius, 1967. Fotografia de A. Kunčius. \versal{LATGA"-A},
  Vilnius, 2013.
\item
  As igrejas Bernardina e de Santa Ana. Fotografia de J. Bułhak, ca.
  1930. Cartão"-postal, cortesia de A. Kubilas.
\item
  Vilna vista por autor anônimo do século 17. Facsímile realizado por
  Barousse, retirado de de J.\,K. Wilczyński, \emph{Album de Wilna},
  Paris, ca. 1850. Museu Lituano de Arte.
\end{enumerate}
