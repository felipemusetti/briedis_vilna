\chapter[Nota sobre a edição, \textit{por Laura Tupe}]{Nota sobre a edição}
% \begin{flushright}
% \textsc{laura tupe}\\
% \textit{Cônsul-geral da Lituânia em São Paulo}
% \end{flushright}

\noindent{}É com grande prazer que, como Cônsul"-geral da República da Lituânia em
São Paulo, apresento ao público lusófono a obra \textit{Vilna, cidade dos
outros}, de autoria do escritor e historiador lituano Laimonas Briedis.
Retratada por grandes personalidades estrangeiras que passaram pela
antiga cidade em diferentes épocas, desde sua fundação há quase 700 anos
até os dias de hoje, a geografia multicultural de Vilna é contada
de maneira peculiar através do contexto histórico da Europa. O que nos
leva a uma profunda e inesperada percepção da história da Lituânia, 
além de demonstrar o espírito fascinante da cidade.

A ideia da edição brasileira de um livro, que é marco internacional no
campo da literatura sobre a Lituânia, surgiu em 2018 durante a primeira
edição do Labas, Festival da Lituânia, na Casa Museu Ema Klabin.
Convidado pelo Consulado Geral da Lituânia em São Paulo, Laimonas
Briedis compartilhou com o público a interessante história de Vilna. É
importante relembrar que no final do século \textsc{xix} e no início do século
\textsc{xx}, o Brasil recebeu mais de 40 mil imigrantes lituanos, e acolhe ainda
hoje número significativo de seus descendentes. Após várias tertúlias
com Laimonas Briedis em São Paulo --- cidade que o interessa não somente
pela possibilidade de apresentar suas conquistas literárias, mas também
pelos ecos da Lituânia dentro da grande diversidade brasileira ---, foi
projetada uma edição em português desta obra, que seria lançada durante
a comemoração da Semana Litvak de São Paulo, ano proclamado pelo
Parlamento da Lituânia como o ``Ano do Gaon de Vilna e da História
dos Judeus da Lituânia''. As palavras de apoio do ilustre e grande amigo
da Lituânia de origem \textit{litvak} --- Celso Lafer, presidente da
Casa Museu Ema Klabin e ex"-ministro das Relações Exteriores do Brasil ---,
incentivaram ainda mais a realização desta ideia que se tornou realidade
graças ao trabalho de mais um amigo da Lituânia, o tradutor Fernando
Klabin, também de origem \textit{litvak}, aos quais transmito nossa imensa
gratidão. Agradeço também à Ayllon Editora, que abraçou nossa ideia
desde a fase inicial do projeto, bem como ao Instituto de Cultura da
Lituânia, pelo apoio financeiro à tradução deste livro.

Infelizmente, a pandemia mundial obrigou"-nos a corrigir os ambiciosos
planos da primeira Semana Litvak de São Paulo, concebidos pelo Consulado
Geral da Lituânia em parceria com várias instituições judaicas de São
Paulo, fazendo"-nos adiar o evento para um futuro mais previsível. Assim
sendo, a edição brasileira do livro de Laimonas Briedis \textit{Vilna, cidade
dos outros} servirá como epílogo brasileiro das comemorações do
Ano do Gaon de Vilna e da História dos Judeus da Lituânia, que
tiveram início na Lituânia no final de novembro de 2019, com a exposição
trazida do Brasil ``Um modernista brasileiro de \textit{Vilnius}: o retorno de
Lasar Segall''.

Mais que uma simples tradução, o livro foi enriquecido por uma
introdução de autoria do próprio autor e pelo prefácio de Celso Lafer, o
que torna a obra ainda mais especial e contextualiza os laços entre
Lituânia e Brasil. Como as edições do livro em inglês, lituano, alemão,
chinês e russo, \textit{Vilna, cidade dos outros} irá deliciar muitos
apreciadores de história e da literatura fina. Permitam"-me desejar"-lhes
uma ótima leitura e convidá"-los a conhecer Vilna --- minha cidade natal,
que amo e admiro!

\begin{flushright}
\textsc{laura tupe}\\
\textit{Cônsul-geral da Lituânia em São Paulo}\\
\medskip
Novembro de 2020
\end{flushright}